\documentclass[11pt]{article}				%General macro to use with document

\pdfpagewidth 8.5in
\pdfpageheight 11in

\usepackage{amssymb}%include the AMS symbols
\usepackage{graphicx}% Include figure files
\usepackage{dcolumn}% Align table columns on decimal point
\usepackage{bm}% bold math
\usepackage{lastpage}
\usepackage{fancyhdr}
\pagestyle{fancy}
\lhead{Team \#2099}
\rhead{Page \thepage of \pageref{LastPage}}

\begin{document}
	\title{Districting Done {\it Right}}
	\author{Team 2099}


\maketitle
	\begin{abstract}
how do we do this front 3/4 page summary??? (need to write it)
	\end{abstract}



\section{Introduction}
In compliance with the United States Constitution and other subsequently
passed laws, the makeup House of Representatives is to closely mirror the 
population distribution of the country. With the exception that all
states must be minimally represented by one person, seats are allocated
such that the highest-population states will control a proportionately
high level of power in the lower chamber. When sending multiple
representatives, it is the individual state's responsibility to apportion 
its population into districts that, though not necessarily non-partisan,
must resemble each other in total population. Malapportionnment, by
enlarging a specific district to decrease its population's influence, is
strictly illegal and unconstitutional.

New York, the second most populous state in the Union, elects bi-annually 
29 delegates to the House, each representing his or her own congressional 
districts of roughly equal population. Within each state, however, there
is no specific requirement of how these districts are drawn, so long as
they remain in general parity regarding the number of inhabitants.
Furthermore, each state is responsible for redrawing their own district
lines as new census data becomes available and certain areas gain or lose 
relative popularity within the state. Alternatively, districts must be
redrawn whenever seats are gained or lost as a result of a population
shift within the country. Predictably, districting is politically a hotly 
contested issue, and, more pertinently, a generally difficult task to
perform by hand.

Historically, districting practices, left up to the state, have allowed
misapportionment to improperly and untruthfully shift political power.
Perhaps the most well known incident centered around Massachusetts
Governor Elbridge Gerry \cite{ref:gerry}, who in the early 1800s approved 
a controversial districting plan which created a “salamander-shaped”
district to group areas of political opposition under one national
representative. This outraged the opposing party, who alleged that he was 
now given unfair advantage. Retroactively named, "gerrymandering" refers
to any districting action which, via shape or size, intentionally seeks
to dilute the relative power of political opposition. See figure
\ref{fig:arizona} for an example of a currently existing, especially
odd-shaped district.

\begin{figure}
\begin{center}
\end{center}
\caption{Arizona's 2nd congressional district, widely thought to be
gerrymandered. (Source: NationalAtlas.gov)}
\label{fig:arizona}
\end{figure}

Amongst political scientists, there does not exist a consensus regarding
the effects gerrymandering \cite{ref:mcdonald}. While there is
mathematical evidence that suggests gerrymandering can give an advantage
to a majority party or philosophy \cite{ref:howto}, it is claimed that
the effects of a successful gerrymander by use of shape in a state of
high contention (and thus a state where a gerrymander would have the most 
effect), any gain would be relatively short lived and would eventually be 
negligible due to changing voting trends \cite{ref:analyzing}. On the
other hand, gerrymandering by skewing size can have tremendous effects to 
suppress the political influence of a group that is tightly contained
geographically, such as the African-American population in the
low-country of South Carolina.

Even within this Mathematical Contest in Modeling team there are
differences in opinion with regard to whether or not districts should be
determined by legislative procedures within a state or an impartial
computer algorithm. Thus the motivation of this paper is not simply to
develop such an algorithm, but also to consider whether or not the
algorithm, if successful, should be employed at all.

This paper will first develop a mathematical model for the districting
problem conceptually as expanding “bubbles” in a fixed volume, with
each bubble containing equal mass (analogous to population). We then
develop an algorithm based on this model, where the bubbles and fixed
volume translate to districts and state borders respectively. Literature
suggests that our algorithm fits well into the multi-start constructive
heuristic classification, but it also contains greedy and probabilistic
elements \cite{ref:overview} \cite{ref:orbook}. While there exist many
heuristic algorithms for solving the districting problem
\cite{ref:heuristic1} \cite{ref:heuristic2} \cite{ref:heuristic3}, our
algorithm was conceived independently and, according to our research,
contains nuances not seen in existing methods. Finally, the
implementation is applied to the state of New York, such that the results 
may be compared actual districts.

\section{Model Fundamentals}
\subsection{Solution Criteria}\label{sec:criteria}
As expected, over the past few decades many cases related to
congressional districting have been heard by the United States Supreme
Court. /cite{watson} Verdicts and decisions arising from these trials
include a strict consideration of malapportionment as illegal: districts
within a state must contain near-equal population, but no other
characteristics have been clearly defined. While the courts do require
that districts should be "reasonably contiguous and compact," this
ambiguous statement is often ignored, as it is not immediately clear how
to define "reasonably" contiguous. As a manner of practice, it is
expected that districts will be drawn in good faith, such that political
gain is not a consideration in the placement of the individual boundaries.

Upon careful consideration, we make the assumption that a "good" and
"fair" districting scheme accounts for three essential criteria: equal
population amongst districts, contiguity, and compactness. The first two
criteria are simple and non-ambiguous, while for the last criterion we
define the following benchmark: $min(\frac{perimeter}{area})$. For a
constant area, the shape that minimizes this simple expression is a
circle, whereas a district shape subject to gerrymandering would have a
much greater ratio. We define population equality and contiguity as our
constraints, while our compactness benchmark defines the quality of a
given solution.

Note that missing from our criteria are population demographics, which
are frequently said to be used by gerrymandering politicians. The absense 
of such traits yields an unbiased districting algorithm.

\subsection{Conceptual Model}
Given our solution criteria specified above, an intuitive, analogous
system was developed that featured expanding bubbles. Such shapes are
obviously contiguous and attain the best "score" in the compactness
benchmark. Thus, the political districting problem can be modelled as
\textit{bubbles} that are expanding inside a fixed area in the shape of
the state border. The area of each bubble represents a single district,
and its mass represents the contained population. As the bubbles expand,
they not only try to retain their shape, but they also inadvertently
deform when they reach a barrier, obstacle, or other bubble. Although all 
bubbles are artificially constrained to hold at any point the same mass,
they will eventually cover the entire area of the state while retaining a 
reasonably contiguous and convex shape.

\begin{figure}
\begin{center}

\end{center}
\caption{Two bubbles in a non-uniform fixed area, with the bottom having
a greater density than the top. For all images, each bubble encompases
equally sized populations. (a) Both bubbles are the same size. (b), (c)
The top bubble expands more quickly than the bottom bubble, since the top 
of the fixed area is less dense. (d) The bubbles begin to deform due to
collisions with the fixed area's boundary and each other. (e) The steady
state when both bubbles have expanded to their maximum size.}
\label{fig:bubble}
\end{figure}

Now consider the case where a bubble's rate of expansion is determined by 
the number of people it takes in. In an area modeled as having a uniform
density, these expanding bubbles would grow at the same rate and, in
their steady state, partition the area into equally sized regions. But a
realistic state, of course, does not have a uniform population density.
If we weight the interoir of the state by population and keep constant
the number of people that each bubble encompasses, then the bubbles would 
grow at varying rates, such that low density areas encourage rapid growth 
and high density areas encourage slow growth. In their steady state, the
bubbles would partition the area into differently sized regions while
maintaining the requirement that they all contain approximately the same
number of people. This concept is illustrated in figure \ref{fig:bubble}. 
The idea that each bubble should 1) grow such that the rate of people
absorbed by the bubble is constant, and 2) try to retain its bubble-like
shape, is what we attempt to enforce in our algorithm.

\section{Details of the Districting Algorithm}

Although, initially, it may seem plausible to run a physics-based
simulation of several bubbles within a somewhat continuous environment,
bubble deformation is extremely difficult to simulate on a large scale.
Thus transforming the conceptual model into an algorithm that can be
realistically implemented was necessary. The expanding bubble concept is
mostly maintained as determined above, but several computational details
and support structures were added as described in the following sections.

	\subsection{Design for "Expanding Bubbles"}\label{sec:bubbles}

The first step in our algorithm is to create an \textit{environment
matrix}, a two dimensional matrix containing data about the state to be
districted, for the bubbles to expand in. We use a matrix of integers to
define the interior area of a state, with other locations in the matrix
being "off limits." Within the state's interior, each matrix element is
assigned a number which represents the population within an equivalent
square area in the state being modeled. Obviously, a higher number
corresponds to a greater population within that unit square. An example
of a simplified, theoretical state represented as a coarse ten by ten
environment matrix is presented in figure \ref{fig:10x10}a. 

A second matrix, which we call the \textit{bubble matrix}, is used to
store the locations of each bubble by means of an index number ($x$)
ranging between one and the desired number of districts. Each bubble is
initialized as a single entry in this new matrix. The start location of
each bubble is determined probabilistically via the values in the
environment matrix, as shown in equation \ref{eq:one}. $P(i,j)$ is the
probability that bubble is placed at location $(i,j)$, $R_{i,j}$ is the
value of the environment matrix at location $(i,j)$, $S$ is the set of
locations that lie within the interior area of the state, and $n$ is a
constant. A larger value of $n$ corresponds with greater skewing of
probability towards more populous areas, and the case $n=0$ corresponds
with every $(i,j) \in S$ having an equal liklihood of being picked. We
desire most bubbles' starting locations to be near population centers
because we make the assumption that these are good approximations for
where most bubbles will end up in an optimal solution.

\begin{equation}\label{eq:one}
P(i,j) = \left\lbrace \begin{array}{cc}
\frac{\displaystyle R_{i,j}^n}{\displaystyle \sum_{(i,j) \in S}
R_{i,j}^n} & \mathrm{for}\, (i,j) \in S \\ \\
0 & \mathrm{for}\, (i,j) \notin S
\end{array}\right.
\end{equation}

Following initialization, the bubble containing the fewest people begins
to expand itself. The size of a bubble is defined as ${F(x)=\sum_{(i,j)
\in B_x}R_{i,j}}$, where $B_x$ is the set of locations in the environment 
matrix (which is determined by the values in the bubble matrix) that
belong to the bubble with index $x$. Let the smallest bubble be $x_{min}$ 
such that ${F(x_{min})={min} F(x)}$ over all $x$. The bubble $x_{min}$ is 
expanded until it is strictly larger than all others, thereby giving the
dubious role of "smallest bubble" to another. The process repeats until
all $(i,j) \in S$ are associated with some bubble.

The primary method by which a bubble grows is to absorb an empty adjacent 
location. Since there are generally numerous possible locations to choose 
from, relative probabilities are assigned to each potential expansion
spot such that the expanding bubble remains as compact as possible (in
open space, this equates to a roughly circular shape). The most probably
location for expansion will be that closest to the bubble's current
center of area, defined as the average position (without regards to
population) of every discrete location contained within the bubble. In
descending order, less appealing expansion spots are assigned
probabilities that decay in a constant manner. Currently, any arbitrary
expansion spot is 50\% more likely to be chosen than its slightly more
distant peer.

In the event that a bubble is not adjacent to any empty locations, but it 
still needs to expand, it performs what the model colloquially refers to
as a \textit{steal}. More specifically, the bubble will assign to itself
a location from a neighboring bubble, thus gaining possession of the
enclosed population. This growing method is also a probabilistic step
that collects possible expansion spots to which relative probabilities
are assigned. Unlike the previous step, however, consideration is given
to the original possessor of any would-be expansion spot. Specifically,
all possible expansion spots are ordered such that steals are most likely 
to occur from the largest neighbors. On multi-pixel borders, the
expansions spots are further ordered by proximity to the expanding
bubble's center of area.

\begin{figure}
\caption{Matricies from a $10 x 10$ districting problem. (a) The
environment matrix. (b) A good result of the algorithm. (c) Another good
result of the algorithm, but a different scheme. (d) A bad result of the
algorithm.}
\label{fig:10x10}
\end{figure}

For the algorithm to end at what we consider a solution, two criteria
must be met: all locations in the interior of the state are associated
with some bubble, and the bubbles have roughly equal size. All bubbles
will first expand until the whole of the state is filled (with perhaps
some stealing taking place), To balance population sizes, they should
continue to steal from each other by again continuously selecting the
smallest bubble/district and expanding it. Note that at this point in the 
algorithm, a given district is expanded until it is 3\% above the total
population of the state divided by the number of districts.

The algorithm boosts small districts to (just over) the desired
population size until each is nearly uniform. For the purposes of this
condition, "nearly uniform" is defined to be the range ${\left[-8\%
,+12\% \right]}$ of the total population divided by the total number of
districts. The range is skewed towards values greater than the optimum
due to algorithm design: although the system can force individual
bubble/district population to rise, it must wait for it to fall. This
fall-time can be extremely lengthy, truly representing the majority of
the computation in the algorithm.

The algorithm, in its original form, is \textit{heuristic} in that, even
though the optimality of solutions cannot be guaranteed, it should still
find good and feasible solutions. It is also \textit{constructive}
because it builds a solution from the ground-up and terminates upon
discovery of one that meets all of the criteria and requirements
explained previously. Furthermore, it exhibits \textit{multistart}
behavior since each time the algorithm is run, the start-locations of
each bubble are determined probabilistically.

To have a reasonable chance of approaching an optimal solution for a
given data set, the algorithm should be run multiple times. The results
should be compared, and the best starting locations noted. To expand upon 
the previous point, the model is certainly \textit{probabilistic}, as it
uses probability at virtually every step. Finally, the model is somewhat
\textit{greedy} in nature because at every step, it strives to expand the 
bubbles in the best, most compact way possible (even though probabilities 
allow for nonoptimal expansions). The tyoes of algorithms mentioned above 
are well studied, but research has not found another algorithm with
precisely this combination of characteristics.

	\subsection{Finding a Reasonably Good Solution}

The result of the algorithm on a simplified case is shown in figure
\ref{fig:10x10}. In this simulation, three districts are planned for a
theoretical state that fits into a $10 x 10$ grid. The state has two
moderate population centers, one in the top left corner and one slightly
below it (look where the values in figure \ref{fig:10x10}a are large).
Rarely, the algorithm might make a poor decision while expanding, as
illustrated by the discontiguous solution in figure \ref{fig:10x10}d.
However, most commonly, the algorithm converges to a continuous and
compact solution similar to the "peace sign" arrangement in figure
\ref{fig:10x10}c. Occasionally, the solution takes a different shape,
like the "horizontal bar" arrangement in figure \ref{fig:10x10}b. It is
obvious that we would want to throw out solutions like \ref{fig:10x10}d,
but less clear is the deciding between \ref{fig:10x10}b and
\ref{fig:10x10}c, both of which appear to be good solutions.

It becomes necessary, then, to develop some sort of quality measure. Of
the three solution criteria described in section \ref{sec:criteria}, only 
population equality is completely garuanteed to be satisfied. Therefore
it makes sense that the benchmark we should use would test contiguity and 
compactness. We consider two: \textit{circular} (equation 1111111) and
\textit{rectangular} (equation 22222222). The former is perimeter divided by
area, where the optimal shape is a circle. The latter is area of the
bubble divided by area contained within the smallest possible rectangular 
cover, where the optimal shape is a rectangle. In both cases, a greater
value implies a more optimal solution. See figure *****need to make
figure***** for an illustration.  

Returning to the results of the $10 x 10$ case, figure *****need to make
table of values***** shows the numerical result of both quality measures
applied to figures \ref{fig:10x10}b-d. While both benchmarks discount
\ref{fig:10x10}d as being a good solution, they don't come to a consensus 
with regard to \ref{fig:10x10}b and \ref{fig:10x10}c. The rectangular
method favors a very rectangular solution in \ref{fig:10x10}b, while the
circular method favors an approximately round solution in
\ref{fig:10x10}c. So while it's still not clear as to which kind of
solution would be better, we can at least discount the pathologically bad 
ones.

*****equations 111111111 and 222222222 go here-ish*****

	\subsection{Implementation Issues}

While the algorithm runs well in the $10 x 10$ case, it runs
significantly slower with larger environment matricies. Because of this
slow runtime with any decently detailed dataset, changes to certain
aspects of our algorithm were needed such that it would run faster.
Previously, the smallest bubble, $x_{min}$, expanded until it surpassed
the size of the next smallest bubble. In the final version of the
algorithm, it expands until it is strictly greater than the largest of
bubbles. Because the bubbles are now no longer expanding at approximately 
the same rate, some irregular shapes can occur. Compactness and contiguity are thus lost in the revised algorithm, 
which by design did not happen in the original algorithm. As a result, a
suplimentary compactness algorithm should be designed to account for
this tradeoff, even though time constraints do not provide for this.
	
\section{Redistricting New York State: A Case Study}

	\subsection{Details of Implementation on Actual Data}

To test the algorithm developed in the previous sections, unbiased
congressional districts are drawn, twenty-nine in all, in New York State. 
Similar to how the state government would proceed, the algorithm uses the 
latest Census data (in this case, from 2000) in determining how district
lines should be drawn. This data, freely available from the New York
State Data Center \cite{ref:census}, is in the form of a
\textit{shapefile}, a commonly used geographical file format devised by
the Environmental Systems Research Institute \cite{ref:shapefile}. The
shapefile is basically a set of connected polygons that represent
\textit{census tracts}, which are county subdivisions that the United
States Census Bureau use to complete the decannual census
\cite{ref:tracts}. The population of each census tract usually varies
between 1,500 and 8,000, having been designed for homogeneity with
respect to certain population characteristics when first delineated. Each 
polygon in the shapefile contains data regarding its total population and 
area, so it was fairly simple to create a new dataset of population
density $\left( \frac{total population}{tract area}\right) $. Using a
program called Thuban \cite{ref:thuban}, population density was rendered
into a color-gradiented graphic. By outputting this image and
subsequently converting color information into a matrix of integers, we
create an environment matrix (see section \ref{sec:bubbles}) which
represents the state of New York. 

Upon careful study of the data, it becomes painfully clear that the
problem of districting New York is not at all an easy one. Of the more
than nineteen million people in New York state, Nearly half (eight
million) reside in New York City, which comprises less than 1\% of the
total land area of the state \cite{ref:pop}. Additionally, this huge
spike in population is concentrated at a geographical bottleneck, making
it difficult for bubbles to grow. For instance, if there were a bubble at 
the end of Long Island needing to expand, it would have to steal from a
nearby New York City bubble, creating an inefficient chain reaction
ending upstate. Furthermore, the addition of just a single pixel to a
bubble in New York City could drastically alter a bubble's size, wrecking 
havoc on the algorithm's ability to find a convergent solution.

	\subsection{Results of the Districting Algorithm on New York State}
In agreement with our prediction of the unwieldy data due to New York State, we have found complications near New York City arising routinely in the running of our code.  Due to the large amount of stealing that takes place, which cuts into the state with the higher population, our initial goals concerning the compactness and contiguity of the districts are sacrificed for the equality of population.  One may ask oneself if this is better or worse than gerrymandering, due to many of the visual similarities between the resulting districts.  One might also conclude that our algorithm results in districts that are worse off than the gerrymandered. This is especially evident considering the large distance between discontiguous portions of some districts. Figures ******************* show some of the better and worse results of our algorithm.


\section{Conclusion}

In this paper we have combined the qualities of several heuristic
algorithms into one for the purpose of political redistricting without
the influence and interference of legislation. While tradeoffs were made in favor of reasonably fast code, the algorithm itself is still good. The following section details ways this districting method might be improved in terms of computational complexity without having to sacrifice the essence of the algorithm.

	\subsection{Further Work}

It is inevitable and unfortunate that, within the span of a ninety-two
hour competition, many ideas are stumbled upon with no time to implement
them. The first of two is an improvement upon our current algorithm
involving environment matricies of varying size, representing the same
state. The algorithm is run on the smallest version, and the result is
used as an initial guess for the next smallest version. The chain
continues until the problem is solved at the desired high quality
resolution. This textit{multiresolution} approach would decrease
computing time without much, if any, loss of accuracy. Additionally, the
algorithm would not lose its probabilistic nature. In fact, if only the
best small resolution results become initial guesses for higher
resolution problems, then the algorithm takes on a \textit{genetic}
quality. That is, it emulates biological evolution as only the good
results are kept, leading to more optimum solutions.

Another avenue of further development involves extending the bubble idea
to a discrete problem. Our implementation of our expanding bubble
algorithm utilizes a discrete approximation of a continuous $\mathbb{R}^2 
\longrightarrow \mathbb{R}$ function of population density. It would
probably be far more accurate and computationally faster to instead
represent each census tract as a single point with a $(x,y)$ location and 
a value equal to the population at that point. Much of the algorithm
would remain the same, in that we still use bubbles, and they are still
placed probabilistically. But rather than actual circular bubbles one can 
see on a map, the growth step would be of a more computational nature:
simply search for the point whose coordinates are mathematically closest. 
This is effectively the same as having a bubble radius and growing it to
the the nearest neighbor location. Probably this variation's biggest
advantage is that it literally assigns census tracts to districts, so it
has a far better correlation to real life than our current algorithm.

	\subsection{Final Thoughts}
Upon final analysis, we have come to the conclusion that our results are unacceptable for practical use.  This is not to say that our algorithm is fundamentally flawed, or wasn't fully thought through.  On the contrary, these results show that the algorithm wasn't able to take the form which it's authors envisioned.  Implementing a discrete version of the model would most certainly yield the needed speed, and would not have the same ill effects our algorithm produced.  This is certainly an area of research that should be pursued.


\begin{thebibliography}{1}

\bibitem{ref:gerry}Hatfield, Mark O, with the Senate Historical Office
(1997). {\em Vice Presidents of the United States, 1789-
1993}. U.S. Government Printing Office: Washington, DC. pp. 63-68.
\bibitem{ref:mcdonald}McDonald, Michael P (2004). A Comparative Analysis
of Redistricting Institutions in the United States, 2001-2002. {\em State 
Politics \& Policy Quarterly}. 4(4):371.
\bibitem{ref:howto}Sherstyuk, Katerina (1998). How to gerrymander: A
formal analysis. {\em Public Choice}. 95:27-49.
\bibitem{ref:analyzing}Morton, Rebecca B (2004). {\em Analyzing
Elections}. forthcoming, W.W. Norton. pp. 135-168.
\bibitem{ref:overview}Silver, E A (2004). An overview of heuristic
solution methods. {\em Journal of the Operational Research Society}.
55:936-956.
\bibitem{ref:orbook}Rardin, Ronald L (1998). {\em Optimization in
Operations Research}. Prentice Hall: Upper Saddle River, NJ. pp. 680-705.
\bibitem{ref:heuristic1}Catalano, M S Fiorenzo; Malucelli, F (2005).
Randomized Heuristic Schemes For The Set Covering Problem. {\em ORSA
Journal on Computing}. 17(1):10.
\bibitem{ref:heuristic2}Bozkaya, Burcin; Erkut, Erhan; Laporte, Gilbert
(2003). A tabu search heuristic and adaptive memory procedure for
political districting. {\em European Journal of Operational Research}.
144:12-26.
\bibitem{ref:heuristic3}Mehrotra, Anuj; Johnson, Ellis L; Nemhauser,
George L (1998). An optimization based heuristic for political
districting. {\em Management Science}. 44(8):1100.
\bibitem{ref:census}New York State Data Center (2000). Content: 2000 Census data and geographic files for New York state at the Census tract level. [online]. Available from: http://205.232.252.81/selectFiles.asp. Accessed 2007 Feb 10.
\bibitem{ref:shapefile}Environmental Systems Research Institute, Inc. (1998). {\em ESRI Shapefile Technical Description}. [online]. Available from: http://www.esri.com/ library/whitepapers/pdfs/shapefile.pdf. Accessed 2007 Feb 12.
\bibitem{ref:tracts}U.S. Census Bureau, Geography Division (2000). {\em Census Tracts and Block Numbering Areas}. [online]. Available from: http:// www.census.gov/geo/www/cen{\_}tract.html. Accessed 2007 Feb 12.
\bibitem{ref:thuban}Thuban Project Team (2004). {\em Thuban 1.0 Documentation}. [online]. Available from: http://thuban.intevation.org/documentation.html. Accessed 2007 Feb 12.
\bibitem{ref:pop}U.S. Census Bureau (2007). {\em State and County QuickFacts}. [online]. Available from: http://quickfacts.census.gov/qfd/states/36/ 3651000.html. Accessed 2007 Feb 12.



\end{thebibliography}



\end{document}